\chapter{Závěr}

Práce zpracovává rešerši na dostupné vývojové platformy v~rámci výuky programování na \legoEV{}.
Z~výsledků rešerše vyplynulo, že nejvhodnější systém pro EV3 je RTOS EV3RT. 
Hlavními důvody jsou výkonost, real-timovost, otevřený zdrojový kód, snadná modifikovatelnost a~dostupnost pro uživatele na všech běžných operačních systémech (Windows, Mac, Linux).

RTOS EV3RT byl výkonnostně porovnán se standardním systémem v~\EVbrick{\it u} rozsáhlou sadou testů.
Výsledky ukazují výrazný výkonnostní rozdíl při provádění jednotlivých činností. 
EV3RT překonává \lego{} systém ve většině oblastí v~rozmezí sto až tisíci násobku.
% Zároveň je u něj zajištěno velmi přesné časování a nedochází u něj k časovým prodlevám 
Z~těchto důvodů byl RTOS EV3RT vybrán jako vhodný systém pro výuku programování. 

Práce se dále věnuje návrhu vývojového prostředí. 
Vybírá požadavky, které budou u~daného prostředí prioritou.
Pro EV3RT práce zpracovává objektově orientované C++~API s~anglickou dokumentací, které pokrývá kompletní funkcionality \legoEV{}. 
API bylo vyvinuto s~důrazem na jednoduchý přechod z~obrázkového LEGO Softwaru a~proto by pro uživatele mělo být snadné začít v~něm programovat.
To se také potvrdilo během testování se studenty.

Byl proveden i~průzkum několika vývojářských editorů. 
Za nejvhodnější pro začátečníky v~programování byl zvolen Visual Studio Code, kvůli svému jednoduchému UI, které zároveň zachovává rozsáhlé možností přizpůsobení, velkou palety doplňků a~kvalitní našeptávání. 

K~celému prostředí je připravená česká dokumentace. 
Pro uživatele je připraven popis C++~API i~vývojářského editoru. 
Pro každou oblast používání je zpracováno přehledné přirovnání jednotlivých funkcí s~LEGO Softwarem.
Žáci SPŠ a~VOŠ Brno, Sokolská již postupně toto prostředí začínají využívat k~programování.
Během testování s~nimi byli odhaleny některé nedostatky v~dokumentaci a~pojmenování metod v~API. 
Jinak si ale studenti prostředí i~API pochvalovali. 
Za jednu hodinu si zvládli projít dokumentaci a~začít programovat. 
Díky podobnosti API s~LEGO Softwarem jim nedělalo problém psát ekvivalentní programy.
Aktuálně probíhají průběžné úpravy dle zpětné vazbu studentů.

Celé prostředí je momentálně připraveno k~plnému využívání s~\legoEV{}. 
I~tak je tu mnoho oblastí, ve kterých jej lze dále rozvíjet. 
Rozpracovány jsou tutoriály se složitějšími úlohami pro studenty, na kterých by se mohli naučit další činnosti (programování více souběžných tasku, vytváření PID regulátorů, rozsáhlejší loggování do souborů).
V~budoucnu je i~plánu přidání další vrstvy v~API pro jednoduché zakomponování činností jako práce se souřadnicemi, plánování trasy nebo herní logika.
